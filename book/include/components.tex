%  Copyright 2020-2024 Robert Bosch GmbH
%
%  Licensed under the Apache License, Version 2.0 (the "License");
%  you may not use this file except in compliance with the License.
%  You may obtain a copy of the License at
%
%      http://www.apache.org/licenses/LICENSE-2.0
%
%  Unless required by applicable law or agreed to in writing, software
%  distributed under the License is distributed on an "AS IS" BASIS,
%  WITHOUT WARRANTIES OR CONDITIONS OF ANY KIND, either express or implied.
%  See the License for the specific language governing permissions and
%  limitations under the License.

\chapter{Components}

Compared with the core \rfwcore, the \rfw\ contains some additional components, that extend the functionality by useful features.
All components are hosted on GitHub. Most of the components can also be used stand-alone (= outside the context of \rfw). But in case of a stand-alone
usage, the user is responsible for the installation and configuration. Further hints can be found in the readme files of the corresponding GitHub repository.

The following is an overview about all additional components, that are part of the \rfw:

\vspace{2ex}

\textbf{1. RobotFramework\_TestsuitesManagement}

The \textbf{RobotFramework\_TestsuitesManagement} enables users to define dynamic configuration values within separate configuration
files in JSON format. These configuration values are available during test execution - but under certain conditions that can be defined by
the user (e.g. to realize a variant handling or to define parameters that are test bench specific).

The \textbf{RobotFramework\_TestsuitesManagement} also provides a version control mechanism to ensure that the developed tests fit to the test environment.

Homepage: \href{https://github.com/test-fullautomation/robotframework-testsuitesmanagement}{RobotFramework\_TestsuitesManagement}

\vspace{2ex}

\textbf{2. JsonPreprocessor}

The \textbf{JsonPreprocessor} provides additional features in JSON files. These features extend the standard JSON format and make it easier
to use JSON files for e.g. configuring tests (see also \textbf{RobotFramework\_TestsuitesManagement}). The additional features are:
\begin{itemize}
   \item Add comments
   \item Let a JSON file import other JSON files (nested imports; e.g. to avoid redundancy)
   \item Allow also pure Python keywords like \pcode{True}, \pcode{False} and \pcode{None}
   \item Define new parameters and overwrite already existing ones
\end{itemize}

Homepage: \href{https://github.com/test-fullautomation/python-jsonpreprocessor}{JsonPreprocessor}

\vspace{2ex}

\textbf{3. QConnectBase}

\textbf{QConnectBase} is a connection testing library for the \rfwcore. It provides a mechanism to handle trace log continously receiving from a connection
(such as Raw TCP, SSH, Serial, etc.) besides sending data back to the other side. It's especially efficient for monitoring the overflowed response trace log
from asynchronous trace systems.

Homepage: \href{https://github.com/test-fullautomation/robotframework-qconnect-base}{QConnectBase}

\vspace{2ex}

\textbf{4. RobotLog2RQM}

\textbf{RobotLog2RQM} writes \rfwcore\ test results (available in XML format) to the IBM® Rational® Quality Manager (RQM).
This includes:
\begin{itemize}
   \item Create all required resources (Test Case Excution Record, Test Case Execution Result, ...) for new testcase on RQM
   \item Link all testcases to provided testplan
   \item Add new test result for existing testcase on RQM
   \item Update existing testcase on RQM
\end{itemize}

Homepage: \href{https://github.com/test-fullautomation/robotframework-robotlog2rqm}{RobotLog2RQM}

\vspace{2ex}

\textbf{5. RobotLog2DB}

\textbf{RobotLog2DB} writes \rfwcore\ test results (available in XML format) to a database. This database is provided by another component
called \textbf{TestResultWebApp}. Based on the content of the database the \textbf{TestResultWebApp} presents an overview about the whole test execution
and details about each test result.

Homepage: \href{https://github.com/test-fullautomation/robotframework-robotlog2db}{RobotLog2DB}

\vspace{2ex}

\textbf{6. PyTestLog2DB}

\textbf{PyTestLog2DB} writes test results of the Python module pytest (available in XML format) to a database. This database is provided by another component
called \textbf{TestResultWebApp}. Based on the content of the database the \textbf{TestResultWebApp} presents an overview about the whole test execution
and details about each test result.

Homepage: \href{https://github.com/test-fullautomation/python-pytestlog2db}{PyTestLog2DB}

\vspace{2ex}

\textbf{7. TestResultWebApp}

\textbf{TestResultWebApp} is a web-based application and is used for visualizing and tracking test execution results. It provides charts from an overview of
the test result to the detail of all included test cases. It also provides tools to control the quality of developing software version (under testing)
by the a graphical comparison of test results from different test executions

Homepage: \href{https://github.com/test-fullautomation/testresultwebapp}{TestResultWebApp}

\vspace{2ex}

\textbf{8. GenPackageDoc}

\textbf{GenPackageDoc} provides a toolchain to generate a documentation in PDF format out of Python sources that are stored within a repository.
The content of this documentation is taken out of the docstrings of functions, classes and their methods. All docstrings have to be written in
reStructuredText (RST) format, that is a certain markdown dialect.
It is possible to extend the documentation by the content of additional files either in RST format or in LaTeX format.
\textbf{GenPackageDoc} is also designed to consider setup informations of a repository.

Homepage: \href{https://github.com/test-fullautomation/python-genpackagedoc}{GenPackageDoc}

\vspace{2ex}

\textbf{9. PythonExtensionsCollection}

The \textbf{PythonExtensionsCollection} extends the functionality of Python by some useful functions that are not available in Python immediately.
This covers for example file and folder operations, string operations like normalizing a path and a pretty print method.
Also a file comparison feature is available.

Homepage: \href{https://github.com/test-fullautomation/python-extensions-collection}{PythonExtensionsCollection}

\vspace{2ex}

\textbf{10. RobotframeworkExtensions}

The \textbf{RobotframeworkExtensions} extend the functionality of the \rfwcore\ by some useful keywords.
This covers for example string operations like normalizing a path and a pretty print keyword (especially for composite
Python data types). The \textbf{RobotframeworkExtensions} keywords are implemented in Python (see \textbf{PythonExtensionsCollection}).

Homepage: \href{https://github.com/test-fullautomation/robotframework-extensions-collection}{RobotframeworkExtensions}

\vspace{2ex}

\textbf{11. Tutorial}

The \rfw\ tutorial contains robot code examples together with the documentation explaining how to use these examples and how they work.
The basic idea behind this tutorial is not to have theory only but additionally to this theory also the possibility to experience what is explained inside.
The main focus of this tutorial is on some feature extensions available for the \rfwcore.

Homepage: \href{https://github.com/test-fullautomation/robotframework-tutorial}{Tutorial}

\newpage

\textbf{12. Development environment}

The \rfw\ installer also installs VSCodium, that is an open source editor. VSCodium is preconfigured especially for the usage together with
the \rfw.

Features:

\begin{itemize}
   \item Workspace with included testcases folder, documentation and tutorial
   \item Previews for certain formats like, MD, RST, PDF, PlantUML
   \item Support of the extended JSON syntax of \textbf{RobotFramework\_TestsuitesManagement}
   \item Static code analysis by Robocop, that is an also included static code checker
\end{itemize}




