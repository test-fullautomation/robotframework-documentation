%  Copyright 2020-2024 Robert Bosch GmbH
%
%  Licensed under the Apache License, Version 2.0 (the "License");
%  you may not use this file except in compliance with the License.
%  You may obtain a copy of the License at
%
%      http://www.apache.org/licenses/LICENSE-2.0
%
%  Unless required by applicable law or agreed to in writing, software
%  distributed under the License is distributed on an "AS IS" BASIS,
%  WITHOUT WARRANTIES OR CONDITIONS OF ANY KIND, either express or implied.
%  See the License for the specific language governing permissions and
%  limitations under the License.

\chapter{Introduction}

The \rfwcore\ is an automation framework that can be used to execute automated software tests. It's open source, keyword driven, and extensible.
Tests are implemented in simple text files.

The extensibility of the \rfwcore\ gives every test developer the possibility to add missing functionality easily. But when thinking about missing
functionality, some aspects urgently need to be considered:

\begin{itemize}
   \item Additions should be designed carefully to make them as much generic as possible. This will make the additions useful for a wide range of test developers.
   \item Additions have to be bundled in a meaningful way, and they have to be set under version control.
   \item Tests need to be reproduced, and therefore also the entire test system (\rfwcore\ together with all additions) must be reproducible.
\end{itemize}

And these deliberations are not only related to the \rfwcore\ itself. Developing software in an efficient way also requires a development environment.
And this development environment most probably needs to be configured - at least to be able to handle the additions also.
It has to be ensured to use open source software only, to avoid any license fee.
And finally it must be ensured that every test developer within a team of developers uses the same development environment with the same settings.

To cover all aspects mentioned above, the following things have been done:

\begin{itemize}
   \item Download a certain Python version (because the \rfwcore\ is a Python based application).
   \item Add further useful Python modules to the Python installation.
   \item Download a certain version of the \rfwcore. This is the base.
   \item Implement (carefully) some changes within the core source code of the \rfwcore.
   \item Add further components to provide useful features that are currently missing in \rfwcore\ (like a management of test configuration values
         and a version control mechanism).
   \item Select a development environment (here: VSCodium).
   \item Configure this development environment to handle especially the \rfwcore\ code together with all additions.
   \item Put all things together in a separate installer.
\end{itemize}

\textit{The outcome is one single installer, that installs all together: the framework, additional libraries, the development environment and all settings.}

\vspace{2ex}

This is what we call "\textbf{\rfw}", with \textbf{AIO} means: \textbf{A}ll \textbf{I}n \textbf{O}ne.

\vspace{2ex}

With this solution the entire test setup is reproducible by every developer on every computer.

And there is no need any more for any user, to install all affected components manually, and there is also no need any more for any user,
to spend effort on the configuration of the test setup. After using the \rfw\ installer, test developers immediately can start to implement tests.

