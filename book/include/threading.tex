%  Copyright 2020-2024 Robert Bosch GmbH
%
%  Licensed under the Apache License, Version 2.0 (the "License");
%  you may not use this file except in compliance with the License.
%  You may obtain a copy of the License at
%
%      http://www.apache.org/licenses/LICENSE-2.0
%
%  Unless required by applicable law or agreed to in writing, software
%  distributed under the License is distributed on an "AS IS" BASIS,
%  WITHOUT WARRANTIES OR CONDITIONS OF ANY KIND, either express or implied.
%  See the License for the specific language governing permissions and
%  limitations under the License.

\chapter{Threading in Robot Framework}

\section{Background}

The Robot Framework core (\rfwcore) now includes advanced threading features. This enhancement enables the simulation of concurrent processes and interactions within test environments.

\section{Threading Implementation}

The threading functionality in Robot Framework introduces the THREAD keyword, which allows for the execution of parallel processes within tests. Here are some key aspects of how THREAD works:

Keyword Usage:
The THREAD keyword functions similarly to FOR, WHILE, or TRY keywords. It can be utilized within a test case or included in a User Defined Keyword. This flexibility allows for a wide range of multi-threaded test scenarios.

Immediate Thread Start:
A thread starts executing immediately after its declaration. This means that as soon as the line THREAD ${ThreadName} ${IsDaemon} is executed, the thread begins its operation. This immediate start is crucial for real-time testing and synchronization.

Variable Scope in Threads:
Each thread has its own variable scope. Variables declared or modified within a thread do not affect the variables in other threads or the main test flow. This separation ensures that threads operate independently and reduces the risk of data conflicts.

\textbf{Threading Syntax:}

\begin{robotcode}
THREAD    ${ThreadName}     ${IsDaemon}
       [Thread's actions and keywords]
END
\end{robotcode}

This syntax is used to define a thread in Robot Framework tests. Here's a breakdown of the parameters:

\rcode{\$\{ThreadName\}}: This is a required parameter that specifies the name of the thread. It's used for identification and logging purposes. The thread name should be unique within the context of the test suite to avoid confusion. Thread name is case-sensitive.

\rcode{\$\{IsDaemon\}}: This is an optional parameter that determines the nature of the thread. When set to true, the thread is treated as a daemon thread. Daemon threads are typically background tasks that exit when all non-daemon threads have completed. If not specified, the thread is considered a non-daemon thread by default.

The use of named threads and the option to set them as daemon threads allow for precise control over the threading behavior in test cases, facilitating complex test scenarios.

\section{Threading Keywords}

\subsection{Send Thread Notification Keyword}

This keyword is utilized to send a notification from one thread to another, either to a specific thread or broadcasted to all threads.

\textbf{Syntax:}

\begin{robotcode}
Send Thread Notification    notification_name    params=None     dst_thread=None
\end{robotcode}

\textbf{Parameters:}

\rcode{notification\_name}: Mandatory. The unique name of the notification to be sent. This name is used by receiving threads to identify the notification.

\rcode{params}: Optional. The payload of the notification. It can be a string or any structured data and is passed to the receiving thread along with the notification. Default is None, which means no payload is sent.

\rcode{dst\_thread}: Optional. The specific target thread to which the notification should be sent. If not specified or set to None, the notification is broadcasted to all threads.

\subsection{Wait Thread Notification Keyword}

Allows a thread to wait for a specific notification, providing a mechanism for synchronization and response to inter-thread communication.

\textbf{Syntax:}

\begin{robotcode}
Wait Thread Notification     notification_name    condition=None     timeout=5
\end{robotcode}

\textbf{Parameters:}

\rcode{notification\_name}: Mandatory. The name of the notification the thread is waiting for. This should match the name given in \rcode{Send Thread Notification}.

\rcode{condition}: Optional. This parameter allows specifying a Python expression that acts as a condition the notification's payload must meet for the notification to be acknowledged. The condition is provided as a Python expression and evaluated at runtime. In this context, the variable \rcode{payloads} refers to the payload of the received notification, and it can be used within the condition to apply filters or checks based on the content of the notification.


For example, setting condition=\$payloads=='test' means that the thread will continue execution only if it receives a notification named \textbf{notification\_name} whose payload is equal to the string "test". This allows for selective synchronization based on the content of notifications.

\textbf{Example:}

In this example, the thread waits for a notification named "DataUpdate" and proceeds only if the payload of this notification is "test".

\begin{robotcode}
Wait Thread Notification      DataUpdate     condition=$payloads=='test'     timeout=10
\end{robotcode}

\rcode{timeout}: Optional. The maximum time in seconds to wait for the notification. If the notification is not received within this period, the keyword will fail. Default is 5 seconds.

\subsection{Thread RLock Acquire Keyword}

This keyword allows a thread to acquire a re-entrant lock (RLock) with a specified name, providing a mechanism for thread synchronization with optional blocking and timeout features.

\textbf{Syntax:}

\begin{robotcode}
Thread RLock Acquire    rlock_name    blocking=True    timeout=-1
\end{robotcode}

\textbf{Parameters:}

\rcode{rlock\_name}: Mandatory. The name of the re-entrant lock to be acquired.

\rcode{blocking}: Optional. Determines whether the thread should wait for the lock to become available if it's currently held by another thread. Default is \textbf{True}, meaning the thread will wait.

\rcode{timeout}: Optional. The maximum time in seconds to wait for the lock if blocking is set to \textbf{True}. The default value is \textbf{-1}, meaning the thread will wait indefinitely.

This keyword will fail if the lock could not be acquired within the specified timeout or if \textbf{blocking} is set to \textbf{False} and the lock is already held by another thread.

\textbf{Examples:}

\begin{robotcode}
Thread RLock Acquire    MyLock
Thread RLock Acquire    MyLock    blocking=False
Thread RLock Acquire    MyLock    blocking=True    timeout=5
\end{robotcode}

\subsection{Thread RLock Release Keyword}

This keyword releases a previously acquired re-entrant lock (RLock) specified by the given lock name.

\textbf{Syntax:}

\begin{robotcode}
Thread RLock Release    rlock_name
\end{robotcode}

\textbf{Parameters:}

\rcode{rlock\_name}: Mandatory. The name of the re-entrant lock to be released.

The lock must have been acquired by the same thread attempting to release it. Releasing a lock not held by the current thread will result in an error.

\textbf{Examples:}

\begin{robotcode}
Thread RLock Release    MyLock
\end{robotcode}

This keyword will fail if the lock cannot be released, such as when the current thread did not acquire the lock or if the lock was never acquired in the first place.


\section{Summary}

With the introduction of threading and the new keywords 
\begin{robotcode}
Send Thread Notification
\end{robotcode}
and 
\begin{robotcode}
Wait Thread Notification
\end{robotcode}
\rfw\ (Robot Framework) now offers advanced multi-threaded testing capabilities. These enhancements allow for more sophisticated, parallel testing scenarios and improve synchronization and communication between threads.

In addition, the keywords 
\begin{robotcode}
Thread RLock Acquire
\end{robotcode}
and 
\begin{robotcode}
Thread RLock Release
\end{robotcode}
provide powerful mechanisms for controlling shared resources between threads using re-entrant locks (RLocks). These keywords enable precise control over concurrent access to resources, ensuring that test cases can safely handle critical sections without risking race conditions or data conflicts.

Together, these new threading features give users the ability to design complex, real-world testing scenarios involving multiple threads, inter-thread communication, and synchronization. This brings greater flexibility and robustness to automated testing in \rfw, making it well-suited for environments where concurrency and parallel execution are required.
