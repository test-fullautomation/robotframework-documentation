%  Copyright 2020-2023 Robert Bosch GmbH
%
%  Licensed under the Apache License, Version 2.0 (the "License");
%  you may not use this file except in compliance with the License.
%  You may obtain a copy of the License at
%
%      http://www.apache.org/licenses/LICENSE-2.0
%
%  Unless required by applicable law or agreed to in writing, software
%  distributed under the License is distributed on an "AS IS" BASIS,
%  WITHOUT WARRANTIES OR CONDITIONS OF ANY KIND, either express or implied.
%  See the License for the specific language governing permissions and
%  limitations under the License.
\chapter{Apertis Pro}
\section{How to use DLT}

\subsection{dlt-daemon on Apertis}
Verify that \textbf{dlt-daemon} has installed on Apertis Pro target or not.\\
\rlog{systemctl status dlt-daemon}\\
\\
In case \textbf{dlt-daemon} is not available, follow below steps to install and
start dlt-daemon service:
\begin{itemize}
   \item Install \textbf{dlt-daemon} package\\
         \rlog{sudo apt install dlt-daemon}
   \item Start \textbf{dlt-daemon} service\\
         \rlog{sudo systemctl start dlt-daemon}
\end{itemize}

\subsection{Apertis Pro firewall configuration}
In order to capture DLT log/trace from DLT client(\textbf{DLT Viewer},
\textbf{DLTConnector}), DLT client has to communicate with Apertis Pro
(TCP/IP protocol) via port \textbf{3490} (as default).\\
So that, this connection should be allowed on Apertis Pro target.\\
\\
Adopt settings of firewall at Apertis Pro:
\begin{itemize}
   \item Add new rule to allow DLT service at port \textbf{3490} (as default)\\
   Edit \rlog{/etc/iptables/rules.v4} file to add below line
   \begin{robotcode}
...
# Accept dlt for development
-A INPUT -p tcp -m state --state NEW -m tcp --dport 3490 -j ACCEPT
...
   \end{robotcode}

   \item Restart the firewall with changed parameters\\
   \rlog{sudo systemctl restart iptables.service}
\end{itemize}

\subsection{DLTSelfTestApp}
\href{https://sourcecode.socialcoding.bosch.com/projects/ROBFW/repos/selftest/browse/helpers/DLT}
{DLTSelfTestApp} is an application which will be run on the Apertis Pro target for
testing the DLT connection between \rfw\ and target.\\
This package is a part of \rfw\ selftest helpers.\\
To install \textbf{DLTSelfTestApp}, download its debian package on Apertis Pro
target then execute the below command.\\
\rlog{sudo dpkg -i <path/to/dltselftestapp_1.0.0_amd64.deb>}\\
\\
\textbf{DLTSelfTestApp} application will be installed in
\rlog{/opt/bosch/robfw/dlt} directory and can be started with below command:\\
\rlog{/opt/bosch/robfw/dlt/DLTSelfTestApp}\\
\\
Welcome log message \rlog{Welcome to RobotFramework AIO DLTSelfTestApp...}
will be sent at application startup.\\
Then the ping log \rlog{ping message from RobotFramework AIO DLTSelfTestApp}
every 5 seconds.\\
\\
\textbf{\underline{DLT command injection:}}\\
To perform the DLT command injection, use below information:
\begin{itemize}
   \item App ID: \textbf{RBFW}
   \item Context ID: \textbf{TEST}
   \item Service ID: \textbf{0x1000}
   \item Data as Textdata
\end{itemize}

DLT log reponse of \textbf{DLTSelfTestApp} will bases on injected command:
\begin{itemize}
   \item \rcode{welcome}: DLT reponse as above welcome message.
   \item \rcode{exit}: DLT reponse as \rlog{Bye...} then \textbf{DLTSelfTestApp}
                       will be terminated.
   \item Other commands: DLT reponse as combination of data and string.\\
   e.g: \rlog{Data: 000000: 77 65 6c 63 6f 6d 65 31 32 31 32 00 xx xx xx xx
              welcome1212}
\end{itemize}

\subsection{QConnectDLTLibrary}
\href{https://sourcecode.socialcoding.bosch.com/projects/ROBFW/repos/robotframework-qconnect-dlt/browse}
{QConnectDLTLibrary} is part of \rfw.
It provides the ability for handling connection to Diagnostic Log and Trace(DLT) Module.
The library support for getting trace message and sending trace command\\
\\
Sample \rfw\ testcase which are using \textbf{QConnectDLTLibrary} to
test DLTSelfTestApp on Apertis Pro target:
\begin{itemize}
   \item Header of a \rfw\ testcase containing common settings (\rcode{*** Settings ***}) like the setups and teardowns,
         the definition of variables (\rcode{*** Variables ***})
         and the definition of keywords (\rcode{*** Keywords ***}):
   \begin{robotcode}
*** Settings ***
Documentation  This is selftest for DLT connection with DLTSelfTestApp
Library     QConnectionLibrary.ConnectionManager
Suite Setup     Open Connection
Suite Teardown  Close Connection

*** Variables ***
${CONNECTION_NAME}  TEST_CONN_DLTSelfTestApp
${DLT_CONNECTION_CONFIG} =  SEPARATOR=
...  {
...      "gen3flex@DLTLSIMWFH": {
...            "target_ip": "127.0.0.1",
...            "target_port": 4490,
...            "mode": 0,
...            "ecu": "ECU1",
...            "com_port": "COM1",
...            "baudrate": 115200,
...            "server_ip": "localhost",
...            "server_port": 1234
...      }
...  }

*** Keywords ***
Close Connection
   disconnect  ${CONNECTION_NAME}
   Log to console    \nDLT connection has been closed!

Open Connection
   ${dlt_config} =    evaluate    json.loads('''${DLT_CONNECTION_CONFIG}''')   json

   connect             conn_name=${CONNECTION_NAME}
   ...                 conn_type=DLT
   ...                 conn_mode=dltconnector
   ...                 conn_conf=${dlt_config}

   Log to console    \nDLT connection has been opened successfully!
   \end{robotcode}

   \item Sample \rfw\ testcase to verify the ping message from DLTSelfTestApp
   \begin{robotcode}
*** Test Cases ***
Match log/trace from DLTSelfTestApp
   [Documentation]   Match log/trace from DLTSelfTestApp
   [Tags]   DLTSelfTestApp
   ${res}=    verify     conn_name=${CONNECTION_NAME}
   ...                   search_pattern=(DLT:0x01.*RBFW.*)
   ...                   timeout=6    # DLTSelfTestApp pings a message every 5 seconds

   # log to console     \n${res}[0]
   # verify that reponse message should contain "Ping" keyword
   Should Match Regexp     ${res}[0]    DLT:0x01.*RBFW.*Ping.*
   \end{robotcode}

   \item Sample \rfw\ testcase to verify command injection with DLTSelfTestApp
   \begin{robotcode}
Command injection with DLTSelfTestApp
   [Documentation]   Get log/trace from DLTSelfTestApp
   [Tags]   DLTSelfTestApp
   ${res}=    verify     conn_name=${CONNECTION_NAME}
   ...                   search_pattern=(DLT:0x01.*RBFW.*Welcome.*)
   ...                   send_cmd=DLT_CALL_SW_INJECTION_ECU ECU1 1000 RBFW TEST 'welcome'

   # log to console     \n${res}[0]

   ${res}=    verify     conn_name=${CONNECTION_NAME}
   ...                   search_pattern=(DLT:0x01.*RBFW.*other_cmd.*)
   ...                   send_cmd=DLT_CALL_SW_INJECTION_ECU ECU1 1000 RBFW TEST 'other_cmd'

   # log to console     \n${res}[0]
   \end{robotcode}
\end{itemize}

Please refer \href{https://sourcecode.socialcoding.bosch.com/projects/ROBFW/repos/robotframework-qconnect-dlt/browse}
{QConnectDLTLibrary repository} for more details about usage and other
example testcase for DLT connection.
